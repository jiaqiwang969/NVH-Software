\documentclass[UTF8,a4paper,12pt]{ctexart}
\usepackage{geometry}
\geometry{a4paper,margin=2.4cm}
\usepackage{amsmath,amssymb}
\usepackage{booktabs}
\usepackage{array}
\usepackage{graphicx}
\usepackage{xcolor}
\usepackage{listings}
\usepackage{hyperref}
\hypersetup{
  colorlinks=true,
  linkcolor=blue,
  urlcolor=blue,
  citecolor=blue
}

\setmainfont{Times New Roman}
\setCJKmainfont[Path=../../]{SimHei}

\lstset{
  basicstyle=\ttfamily\footnotesize,
  backgroundcolor=\color[gray]{0.97},
  frame=single,
  breaklines=true,
  columns=fullflexible,
  keywordstyle=\bfseries,
  commentstyle=\color{gray}
}

\title{Tang Nano 9K FPGA 冲击检测方案书}
\author{NVH 团队}
\date{\today}

\begin{document}
\maketitle
\tableofcontents

\section{项目目标}
\begin{itemize}
  \item 将软件侧已验证的冲击检测方案(包络+窄带 9.1--9.5 kHz、Teager-Kaiser 能量、峭度等特征)固化到 Tang Nano 9K FPGA,实现低延迟、低功耗在线检测。
  \item 对硬件选型、接口、算法实现和资源优化给出可落地的设计,并明确软件侧的验证与联调路径,为后续在 FPGA/SoC 上量产或迁移到 ASIC 做准备。
\end{itemize}

\section{硬件平台与接口}
\subsection{Tang Nano 9K 核心资源(简要)}
\begin{center}
  \begin{tabular}{p{3.2cm}p{9cm}}
    \toprule
    关键资源 & 说明(以官方手册为准)\\
    \midrule
    FPGA 核心 & Gowin GW1NR-9C,约 9k LUT,约 7k FF,BRAM \textasciitilde0.5 Mb,\textasciitilde20 个 18$\times$18 DSP 块\\
    片上存储 & 1$\times$64 Mbit PSRAM(约 8 MB) \\
    Flash & 1$\times$64 Mbit Nor Flash,用于位流与少量参数存储 \\
    IO/扩展 & 1 个 Type-C(供电/调试),多组 PMOD/排针,可提供 1.8V/3.3V IO 视具体 Bank 供电 \\
    时钟 & 板载 27 MHz 晶振,可用 PLL 派生音频/传感器时钟 \\
    \bottomrule
  \end{tabular}
\end{center}

\subsection{传感器与前端}
\begin{itemize}
  \item 骨传导传感器(UPM01D)输出为 PDM(数字脉冲密度),典型工作电压 1.8 V,时钟 0.4--1 MHz,最大带宽约 11 kHz。
  \item 开发板本身无模拟前端/ADC,需在外部提供:\textbf{PDM 时钟与数据输入}。若改用模拟传感器,则需外接 ADC(Analog-to-Digital Converter,模数转换器)。
  \item IO 电平:PDM 需接到 1.8 V Bank,确保供电与时序满足传感器数据手册要求,走线尽量短并加去耦。
\end{itemize}

\subsection{推荐外设组合}
\begin{itemize}
  \item \textbf{PDM 接口方案(首选)}:直接在 FPGA 侧生成 PDM 时钟(0.8--1 MHz),采集 DATA,引入 CIC+FIR 做抽取到 48 kHz PCM,后续进入检测链路。
  \item \textbf{模拟方案(备选)}:若换为模拟加速度计/麦克风,可选 16--24 bit、采样 48--192 kHz 的音频 ADC(I\textsuperscript{2}S/TDM)或 500 ksps 以上的 SPI SAR ADC。模拟前端需包含抗混叠低通与 9 kHz 邻域的有源带通/增益调理。
\end{itemize}

\section{信号链总体设计}
\begin{enumerate}
  \item \textbf{采样与同步}:PDM 采集 $\rightarrow$ CIC 抽取(64 或 128 倍)$\rightarrow$ 补偿 FIR,得到 48 kHz PCM。
  \item \textbf{窄带带通(9.1--9.5 kHz)}:
  \begin{itemize}
    \item 推荐两个二阶 IIR Biquad 级联(4 阶),Q 适中,系数量化 18 bit;或使用 64--96 阶对称 FIR,利用系数对称性节省乘法器。
    \item 系数在 Python/Matlab 设计,输出固定小数(如 Q1.15/Q2.14),同时导出 C/Verilog 常量表。
  \end{itemize}
  \item \textbf{包络检波}(效果最佳的现有特征):全波整流后用一阶或二阶 IIR 低通(截止 200--500 Hz)平滑,输出包络峰值用于主判据。
  \item \textbf{时域特征扩展}:
  \begin{itemize}
    \item Teager-Kaiser 能量:$\psi[n] = x[n]^2 - x[n-1]\cdot x[n+1]$,再做滑动均值。
    \item 峭度/峭度因子(可选):窗口内 $E[x^4]/(E[x^2])^2$,窗口长度 32--64 点以减轻资源占用。
    \item RMS、峰峰值、包络上升沿时间等作为辅助判据,均采用定点实现。
  \end{itemize}
  \item \textbf{判决逻辑}:
  \begin{itemize}
    \item 双阈值:包络峰值 $>$ 阈值\textsubscript{1},且 Teager 均值 $>$ 阈值\textsubscript{2};支持自适应噪声底线(滑动基线 + 偏移)。
    \item 触发保持/死区:在 $N$ 个采样周期内保持“触发”状态,避免抖动;死区时间控制重复触发。
  \end{itemize}
  \item \textbf{输出与接口}:触发标志、中间特征值可通过寄存器、UART 或 PMOD 输出;调试阶段建议带一组特征 FIFO 供逻辑分析仪抓取。
\end{enumerate}

\section{FPGA 模块划分与实现要点}
\begin{itemize}
  \item \textbf{PDM 采集与解码}:时钟发生器、边沿对齐、CIC(3--5 级)+ 补偿 FIR。CIC 可用加法器链,无 DSP 资源,补偿 FIR 使用系数对称减少乘法器。
  \item \textbf{带通滤波}:若用 IIR,2 个 Biquad 共 10 个乘法器(可映射到 10 个 DSP 块);若用 FIR(64 阶),只需 32 个乘法器并可时分复用 4--8 路乘法器以减资源。
  \item \textbf{包络}:绝对值/平方 $\rightarrow$ IIR 低通(1 个乘法器)。为了抑制直流漂移,可在低通前做一个极低频高通(如 20 Hz)。
  \item \textbf{Teager-Kaiser}:需要 3 个乘法器;使用移位寄存器延时 1/--1 样本,结果进入滑动均值(积分器+减法器实现)。
  \item \textbf{窗口累积与阈值寄存器}:BRAM/分布式 RAM 存窗口;所有阈值、窗口长度、死区时间做可配置寄存器,便于现场调参。
  \item \textbf{流水线与定点}:统一采用 Q1.15 或 Q2.14,保持乘法后做 1--2 bit Guard,末级按需要饱和/截断;滤波器各级加同步对齐信号,保证判决窗口对齐。
\end{itemize}

\begin{lstlisting}[language=C,caption={FPGA 侧冲击检测核心流程(伪代码)}]
// 输入为抽取后的 PCM(48 kHz)
for each sample x[n]:
    bp = bandpass_9k(x[n])               // IIR or FIR
    env = lp_env(abs(bp))                // 包络主特征
    tkeo = bp^2 - x[n-1]*x[n+1]          // Teager 能量
    tkeo_ma = movavg(tkeo)               // 窗口均值
    kurt = kurtosis_window(bp)           // 可选:窗口峭度

    if env > th_env && tkeo_ma > th_tkeo:
        trigger = 1
    else
        trigger = 0

    // 输出:触发状态 + 关键特征,供上位机/MCU 读取
\end{lstlisting}

\section{资源与优化建议}
\begin{itemize}
  \item \textbf{乘法器预算}(48 kHz 情况):CIC 0,补偿 FIR 8--16,带通 IIR 10 或 FIR 8--16(若时分复用),包络 1,Teager 3,RMS/峭度 4--6,总计 $\lesssim$ 20 个 DSP 块即可满足。
  \item \textbf{存储预算}:窗口长度 64--256 点时,BRAM 占用很小(几 KB 级)。可将配置寄存器与 FIFO 放在同一 BRAM 以简化布线。
  \item \textbf{时钟域}:建议单时钟(如 48 MHz)驱动全部逻辑,音频采样 48 kHz 通过时钟使能节拍。PDM 时钟与主时钟分域,PDM 数据通过同步器跨域。
  \item \textbf{功耗/低资源模式}:可只保留包络+Teager 判决,关闭峭度、峰峰值等,满足最小资源占用的 FPGA/MCU 协处理情景。
\end{itemize}

\section{软件侧验证计划}
\subsection{离线算法与定点对齐}
\begin{itemize}
  \item 在 Python/Matlab 中构建浮点参考链路(PDM 解码 $\rightarrow$ 带通 $\rightarrow$ 包络/Teager/峭度),导出测试向量(CSV/NPY)。
  \item 建立定点仿真(与 FPGA 相同 Q 格式、饱和/截断策略),计算误差指标(幅频偏差、包络峰值偏差、触发延迟)。
  \item 对 9.1--9.5 kHz 窄带的滤波器在软件端扫频,验证 3 dB 带宽、群延迟,并与 FPGA 仿真波形对齐。
\end{itemize}

\subsection{仿真与联调}
\begin{itemize}
  \item \textbf{RTL 仿真}:用 Python 生成 PDM/PCM 向量,驱动 Verilog testbench,比较特征/触发标志,与定点模型比对。
  \item \textbf{板级验证}:在 FPGA 上输出实时包络/特征到 UART/逻辑分析仪,使用已知冲击样本回放(软件播放或信号发生器)验证阈值。
  \item \textbf{回归集}:包含低信噪、温度变化、不同佩戴/耦合压力的样本,至少覆盖 20+ 组冲击与若干负样本,统计 FAR/FRR。
\end{itemize}

\section{推进路线与交付物}
\begin{itemize}
  \item \textbf{阶段 1}(软件对齐):完成滤波器设计、定点模型与测试向量;确定阈值范围。
  \item \textbf{阶段 2}(RTL 验证):完成 PDM/CIC/FIR、带通、包络、Teager、判决模块 RTL,仿真通过。
  \item \textbf{阶段 3}(板级联调):在 Tang Nano 9K 上跑通,实测延迟、资源、功耗,调整阈值与死区。
  \item \textbf{阶段 4}(固化与发布):锁定参数,输出 bitstream、寄存器说明、回归报告,为后续 FPGA/ASIC 迁移做准备。
\end{itemize}

\section{构建与使用}
\begin{itemize}
  \item 编译:进入 \texttt{doc/fpga-proposal} 目录,使用 \texttt{xelatex main.tex} 或 \texttt{latexmk -xelatex main.tex} 生成 PDF。
  \item 字体:文档默认使用仓库根目录的 \texttt{SimHei.ttf}(通过 \texttt{Path=../../} 引用);如需更换,可在导言区调整 \texttt{\textbackslash setCJKmainfont}。
  \item 输出物:PDF 方案书(给算法/硬件/验证协作),后续可在同目录增补示意图、时序图与接口表格。
\end{itemize}

\end{document}
